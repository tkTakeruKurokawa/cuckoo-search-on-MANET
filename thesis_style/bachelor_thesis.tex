%
% $Id: thesis_sample.tex,v 1.9 2011/12/08 02:48:59 fukuyasu Exp $
%
\documentclass[11pt]{jreport}
\usepackage{ksu_cse_thesis}
\usepackage{indentfirst}
%\usepackage{graphicx}  % ←graphicx.styを用いてEPSを取り込む場合有効にする
			% 他のパッケージ・スタイルを使う場合には適宜追加

%%%%%%%%%%%%%%%%%%%%%%%%%%%%%%%%%%%%%%%%%%%%%%%%%%%%%%%%%%%%%%%%%%%%%%%%

%%
%% 主に表紙を作成するための情報
%%

%%  タイトル(修論の場合は英語表記も指定)
\title{カッコウ探索を用いた\\
       アドホックネットワーク上の複製配置}
%% 英文タイトル(修論では必須)       
%\etitle{Replication Placement on Ad-hoc Network\\
	% Using Cuckoo Search}

%%  著者名(修論の場合は英語表記も指定)
\author{黒川 岳児}
%%英語著者名(修論では必須)
%\eauthor{Kentaro Hayashibara}

%%指導教員名
\supervisor{林原 尚浩}
%%英語指導教員名(修論では必須)
%\esupervisor{Naohiro Hayashibara}

%% 卒業論文・修士論文(以下のどちらかを選択)
\bachelar	% 卒業論文(4年生用)
%\master  	% 修士論文(M2用)

%%  学科・クラスタ
\department{コンピュータサイエンス}
%\department{ネットワークメディア}
%\department{インテリジェントシステム}

%%  学籍番号
\studentid{544520}

%%  卒業年度
\gyear{2018}		% 提出年が2011年なら,2010年度

%%  論文提出日
\date{2019年1月23日}	% 修士の場合は月(2011年2月等)までとし,英語表記も指定
%\edate{February 2011}


%%%%%%%%%%%%%%%%%%%%%%%%%%%%%%%%%%%%%%%%%%%%%%%%%%%%%%%%%%%%%%%%%%%%%%%%

\begin{document}

\maketitle

%%
%%  概要
%%
\begin{abstract}
本稿では,
「京都産業大学コンピュータ理工学部卒業論文/
  大学院先端情報学研究科修士論文用スタイルファイル」
を用いて卒業論文を作成する方法を解説する.
本稿自身,
「京都産業大学コンピュータ理工学部卒業論文/
  大学院先端情報学研究科修士論文用スタイルファイル」
を用いて記述されており,例によってその使い方を示している.
「京都産業大学コンピュータ理工学部卒業論文/
  大学院先端情報学研究科修士論文用スタイルファイル」
では,タイトルページ,概要,目次,参考文献などの書式を設定している.

  「京都産業大学コンピュータ理工学部卒業論文/
  大学院先端情報学研究科修士論文用スタイルファイル」
は,
\begin{quote}
  \begin{description}
    \item[\tt ksu\_cse\_thesis.sty:] 卒業/修士論文用スタイルファイル
    \item[\tt bachelor\_thesis.tex:] 卒業論文記述例
    \item[\tt master\_thesis.tex:] 修士論文記述例    
  \end{description}
\end{quote}
からなる.

なお,この卒業論文用スタイルファイル(\TeX 版)に関する質問は,
メールにて
\begin{quote}
naohaya@cse.kyoto-su.ac.jp
\end{quote}
まで.

\end{abstract}

%%  目次
\tableofcontents

%%  図目次 (図目次をいれたければ以下のコメントをはずす)
%\listoffigures

%%  表目次 (表目次をいれたければ以下のコメントをはずす)
%\listoftables

\newpage
\pagenumbering{arabic}	% 以降のページ番号を算用数字に

%%%%%%%%%%%%%%%%%%%%%%%%%%%%%%%%%%%%%%%%%%%%%%%%%%%%%%%%%%%%%%%%%%%%%%%%

%%
%%  本文はここから
%%

\chapter{はじめに}
\section{背景}
P2Pネットワークでは,参加ピア間でのデータの共有,耐故障性の向上のため,複数のピアにデータを配置する等,幅広い用途で使用されている.このP2Pネットワークで扱われるデータの需要は一様ではなく、データごとに大きく異なる可能性がある。

\section{問題点}

\section{目的}

\section{論文構成}

\begin{itemize}
  \item タイトル
  \item 著者名
  \item 学士(4年)/修士(M2)の設定
  \item 学科名/クラスタ名
  \item 学籍番号
  \item 卒業年度
  \item 論文提出日
\end{itemize}
である.
これらのデータは,\verb|\maketitle|によってタイトルページに出力される.
また,概要の部分において,論文の内容をまとめる.その内容は論文の2ペー
ジ目(タイトルページの次)に出力される.
このソースでは,目次(\verb|\tableofcontents|)を出力している.
他に,図目次(\verb|\listoffigures|),表目次(\verb|\listoftables|)を
出力することもできるので,必要ならばそれぞれのコメントをはずす.
図目次,表目次については,第\ref{chap:fig-tab-exp}章において説明する.

\section{タイトル}
\subsection{title}
論文のタイトルを記述する.

\section{著者}
\subsection{author}
著者名を記述する.

\subsection{bachelar/master}
卒業論文の場合には,\verb|\master|をコメントアウトし,
\verb|\bachelar|を設定する.
修士論文の場合には,\verb|\bachelar|をコメントアウトし,
\verb|\master|を設定する.

\subsection{department}
所属学科を記述する.
コンピュータサイエンス学科所属の場合には``コンピュータサイエンス''と記述する.

\subsection{studentid}
学籍番号を記述する.

\subsection{supervisor}
指導教員名を記述する.

\subsection{gyear}
卒業年度を記述する.

\section{提出日}
\subsection{date}
論文提出日を記述する.

\chapter{既存の検索・複製手法}
本章では,非構造P2Pネットワークの既存のデータ検索手法,複製手法の説明を行い,問題点を提示する.
\section{検索手法}
検索手法は,目的のデータを持つピアを発見するため,データを要求するピアが送信する検索要求メッセージの転送手法を指す.以下の検索手法が提案されている.\cite{qlv}
\begin{itemize}
	\item {\tt Flooading}
	\par データ要求するピアは,自身の全てのの隣接ピアに検索要求メッセージを転送する.検索要求メッセージを受け取ったピアは,該当データを持っていない場合,同様に検索要求メッセージを自身の全ての隣接ピアに検索要求メッセージを転送する.
	\par 検索要求メッセージにはTTL(Time To Live)が設けられており,メッセージが隣接ピアに転送されるごとに値が1つ減る.値が0になった場合は,データを発見することができなかったと判断され,検索失敗となる.
	
	\item {\tt Expanding Ring}
	\par TTLの値を小さく設定し,Floodingによる検索を開始する.検索に失敗した場合,TTLの値を増加させ,再びFlooadingによる検索を行う.
	\item {\tt k-walker random walk}
\end{itemize}

\section{複製手法}

\chapter{関連研究}
本章では,\cite{kageyama}によって提案された手法の説明と問題点の提示を行う.
\section{提案手法}
\section{問題点}

\chapter{ネットワークモデル}
本章では,本研究に用いるアドホックネットワーク
\section{アドホックネットワーク}
\section{ランダムジオメトリックグラフ}

\chapter{カッコウ探索による複製配置}
\section{カッコウ探索}
\section{Levy walk}
\section{ランダムジオメトリックグラフ上のLevy walk}

\chapter{シミュレーション}
\section{概要}
\section{環境}
\section{結果}
\section{考察}

\chapter{まとめと今後の課題}

\chapter{図,表,数式}\label{chap:fig-tab-exp}

論文では,図,表,数式などを効果的に使用する.

\section{図}

{\tt figure}環境を利用することによって図にキャプション
(\verb|\caption|)を付けることができる.図に付けられたキャプションは
\verb|\listoffigures|によって図目次として出力される.図には章ごとに通
し番号が付けられ,キャプションに\verb|\label|を設定しておくと,
``図\ref{fig:sample}''のように\verb|\ref|によって図を番号で参照するこ
とができる.図\ref{fig:sample}に{\tt figure}環境を用いた記述例を示す.

\begin{figure}
  \begin{center}
    ここで図を取り込む.
    % 試しに,tiger.psが自分のマシンのどこに格納されているかを調べて
    % 以下の命令を有効にしてみて下さい.
    % ただし,同時に\begin{document}より前にある\usepackage{graphicx}
    % も有効にする必要があります.
    % 以下の例ではついでに四角で囲っています.
    %\framebox{\includegraphics[width=5cm,clip]{/usr/local/share/ghostscript/7.07/examples/tiger.ps}}
  \end{center}
  \caption{図の例}
  \label{fig:sample}
\end{figure}

また,{\tt graphicx.sty}などのスタイルファイルを利用することによって
EPS形式の図を文章の中に取り込むことができる.
この場合,\verb|\begin{document}|の前に\verb|\usepackage{graphicx}|を
追加する.

\section{表}

{\tt table}環境を利用することによって図と同じように,キャプションをつ
けたり,ラベルにより参照したりすることができる.また
\verb|\listoftables|によって表目次として出力される.
表\ref{tab:sample}に{\tt table}環境で作成した表を示す.

\begin{table}
  \caption{表の例}
  \label{tab:sample}
  \begin{center}
    \begin{tabular}{|c|c|c|}
      \hline
      8 & 3 & 4\\
      \hline
      1 & 5 & 9 \\
      \hline
      6 & 7 & 2 \\
      \hline
    \end{tabular}
  \end{center}
\end{table}

\section{数式}

\TeX では数式のための機能が豊富である.
{\tt equation}環境などを利用することによって数式に番号を付けることがで
きる.図や表と同じくラベルを付けておけば,``式\ref{exp:sample}''のよう
に数式を番号で参照することができる.

\begin{equation}
  y = ax^2 + bx + c \label{exp:sample}
\end{equation}

\chapter{参考文献}

文献を参照する場合には,論文の最後に参考文献として列挙するとともに,
\verb|\cite|を使って,例えば,
\begin{quote}
  文献\cite{latex}によれば…
\end{quote}
や,
\begin{quote}
  …である\cite{latex2e}.
\end{quote}
のように参照する.

文献の列挙には,{\tt thebibliography}環境などを用いる\footnote{使い方
は,この資料のソースを参照.}.

%% \chapter{ボリューム確認用ページ}

%% 次のページでは,ページあたりのボリュームを確認するために無駄な文章が延々
%% と続いています.

%% \newpage

%% 1行当たりの文字数は…\\
%% 1234567890
%% 1234567890
%% 1234567890
%% 1234567890
%% 1234567890


%%%%%%%%%%%%%%%%%%%%%%%%%%%%%%%%%%%%%%%%%%%%%%%%%%%%%%%%%%%%%%%%%%%%%%%%

%%
%% 謝辞
%%
%% \begin{acknowledgements}
%% 感謝します.
%% \end{acknowledgements}

%%%%%%%%%%%%%%%%%%%%%%%%%%%%%%%%%%%%%%%%%%%%%%%%%%%%%%%%%%%%%%%%%%%%%%%%

%%
%% 参考文献
%%
\begin{thebibliography}{99}
\bibitem{ksu_thesis}
    京都産業大学コンピュータ理工学部卒業論文/大学院システム工学研究科修士論文用スタイルファイル,
    ``{\tt http://rudds.kyoto-su.ac.jp/\symbol{"7E}naohaya/}''.
\bibitem{latex}
    奥村晴彦 著,\LaTeX 入門---美文書作成のポイント---,技術評論社,1993.
\bibitem{latex2e}
    奥村晴彦 著,[改定第3版] \LaTeXe~美文書作成入門,技術評論社,2004.
\bibitem{texbasic}
    OfficeMASA,神代英俊,長島秀行 著,\TeX の基礎,
    ソフトバンクパブリッシング,2002.
\bibitem{latexcomp}
    M. Goossens, F. Mittelbach, A. Samarin 共著,
    アスキー書籍編集部 監訳,The \LaTeX コンパニオン,アスキー出版局,1998.
\bibitem{qlv}
 \bibitem{kageyama}
\end{thebibliography}

%%%%%%%%%%%%%%%%%%%%%%%%%%%%%%%%%%%%%%%%%%%%%%%%%%%%%%%%%%%%%%%%%%%%%%%%

%%
%% 付録
%%
\appendix

\chapter{サンプルプログラム}

プログラムリストや実行結果など,本論を補足する上で必要と思われるものが
あれば付録として付ける.

{
\footnotesize
\begin{verbatim}
#include <stdio.h>
int main(void)
{
    printf("Hello, World!\n");
    return 0;
}
\end{verbatim}
}

%%%%%%%%%%%%%%%%%%%%%%%%%%%%%%%%%%%%%%%%%%%%%%%%%%%%%%%%%%%%%%%%%%%%%%%%

\end{document}
